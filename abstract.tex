\begin{abstract}
\section{Motivation:}
Capillary electrophoresis (CE) is a powerful approach for structural analysis of nucleic acids, with recent high-throughput variants enabling three-dimensional RNA modeling and the discovery of new rules for RNA structure design. Among the steps composing CE analysis, the process of finding each band in an electrophoretic trace and mapping it to a position in the nucleic acid sequence has required significant manual inspection and remains the most time-consuming and error-prone step. The few available tools seeking to automate this band annotation have achieved limited accuracy and have not taken advantage of information across dozens of profiles routinely acquired in high-throughput measurements.

\section{Results:}
We present a dynamic-programming based approach to automate band annotation for high-throughput capillary electro-phoresis. The approach is uniquely able to define and optimize a robust target function that takes into account multiple CE profiles (sequencing ladders, different chemical probes, different mutants) collected for the RNA. Over a large benchmark of multi-profile data sets for biological RNAs and designed RNAs from the EteRNA project, the method outperforms prior tools (QuSHAPE, FAST) significantly in terms of accuracy compared to gold-standard manual annotations. The amount of computation required is reasonable at a few seconds per data set. We also introduce an `$\escore$-score' metric to automatically assess the reliability of the band annotation and show it to be practically useful in flagging uncertainties in band annotation for further inspection.

\section{Availability:}
The implementation of the proposed algorithm as well as the original algorithm is included in the HiTRACE software, freely available as an online server and for download at http://hitrace.stanford.edu.
\section{Contact:} \href{sryoon@snu.ac.kr, rhiju@stanford.edu}{sryoon@snu.ac.kr, rhiju@stanford.edu}
\end{abstract}


%\vspace{1em}\noindent
%\textbf{Keywords:} capillary electrophoresis; alignment; peak fitting; bioinformatics; signal processing
%
%\vspace{1em}\noindent
%\textbf{Running title:} HiTRACE: analysis for capillary electrophoresis
