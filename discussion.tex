%\emph{ WE SHOULD HAVE A SECTION EXPLAINING THE POWER OF USING MULTIPLE TRACES TO DERIVE A CONSENSUS BAND ANNOTATION-- RHIJU}

%\subsection{Use of RNA secondary structure information for band annotation}
%\emph{ I WOULD DE-EMPHASIZE THIS -- PERHAPS NOT EVN MAKE IT A SEPARATE SECTION. WHAT IF YO UHAVE AN RNA WITH UNKNWON SECONDARY STRUCTURE? OR WITH MULTIPLE SECONDARY STRUCTURES?}
%In the band annotation procedure for an RNA sequence, the proposed method constructs the band prediction matrix $\mathbf{P}$ based on the RNA's secondary structure predicted by the Vienna RNA package~\citep{hofacker2003vienna}. Although the secondary structure prediction software typically matches most experimental profiles closely, there may exist cases (\eg, complicated pseudoknots) in which the prediction quality is low or even fails. In such cases, using secondary structure information may lead to incorrect band annotation, but through preliminary filtering we can reduce the possibility of inaccurate annotation.

The proposed method differs from the earlier tools such as QuShape in that it takes account for every profile altogether, whereas other methods focused on a single profile at a time with a reference profile if needed; the distinctive robustness of the proposed method is primarily attributed to this capability to make use of the relationships between profiles. The use of multiple profiles, however, is based on the assumption that all profiles are properly aligned in the preceding step. The band annotation through the proposed method can be less reliable if an accurate alignment of profiles is not warranted, and even can be inferior to the earlier methods if profiles are seriously misaligned. Dealing with such situations would be another challenging task and also a great upgrade for our work, which we set aside as our future work.

$\escore$-score proved useful invariantly to the characteristics of data as illustrated in the result section. According to our experiences, given any data set for CE analysis, the band annotations with $\escore > 0.95$ are almost always reliable and can be safely adopted for the next process in CE analysis whereas the results with $\escore \le 0.95$ are relatively unreliable. In the future, we hope to further reinforce $\escore$-score either by incorporating a statistical approach or by presenting more evidence strong enough for an empirical conclusion.

The proposed algorithm has order of $NK$ time and space complexity, and the practical time demand of band annotation was tolerable in our experiments. The proposed method was implemented in the MATLAB programming environment (The MatheWorks, http://www.mathworks.com), and under the experimental setup used (sequential execution on a Intel core i5 4570 processor with 8-GB main memory), the total time demand of annotating bands in all the 95 data sets did not exceed 4 min (for each data set, mean 2.2837 sec; median 2.2707 sec).

\begin{comment}
\subsection{Use of multiple profiles for band annotation}
The proposed method differs from the earlier tools, FAST, ShapeFinder, and QuShape in that it takes account for every profile altogether, whereas previous methods focused on a single profile at a time with a reference profile if needed; the distinctive robustness of our new method is primarily attributed to this capability to make use of the relationships between profiles. The use of multiple profiles, however, is based on the assumption that all profiles are properly aligned in the previous step. The band annotation through the proposed method can be less reliable if an accurate alignment of profiles is not warranted, and even can be inferior to the earlier methods if profiles are seriously misaligned. The case of L-21 illustrated how an imperfectly aligned set of profiles can lead to an unsatisfactory band annotations as well as $\escore$-score's misjudgement on their quality; for L-21 data sets, $\escore$-score was high since the band annotation on primary profile was exact, but MSE was high because the annotation was inaccurate on the other profiles which were not properly alined with the primary profile.  Dealing with such irregularities would be another challenging task and also a great upgrade for our work, which we set aside for now as our future work.

\subsection{$\escore$-score}
As illustrated by the results from the original 95 and tens of additional data sets, $\escore$-score proved useful invariantly to the charateristics of data. According to our experiences, given any data set for CE analysis, the band annotations with $\escore > 0.95$ are almost always reliable and can be safely adopted for the next process in CE analysis whereas the results with $\escore \le 0.95$ are relatively unreliable. Regardless of its success in sifting out unqualified annotations, however, $\escore$-score did not derive from statistical models and methods, thus has limitations in providing a statistical foundation for its correlation with MSE. Moreover, the amount of evidence gathered from experiments and tests is not sufficient to conclude that the $\escore$-score value of 0.95 can be a definite border of qualification. In the future, we hope to reinforce $\escore$-score either by incorporating a statistical approach or by presenting more evidence strong enough for an empirical conclusion.
%However, the value of 0.95 may need to vary if we perform our method to other data sets with different characteristic; that is, $\escore$-score may be subtly different in its meaning between different types of data sets, although higher $\escore$-score must be consistantly good for all data sets. In the future, we hope to develop the current $\escore$-score into a universal measure which is compatible with any types of data sets, like P-value.


\subsection{Algorithm complexity and time demand of band annotation}
The proposed algorithm relies on dynamic programming and a simple implementation would have the worst-case time and space complexity cubic to the input size. Even so, the practical time demand of band annotation was tolerable in our experiments. Under the experimental setup used (sequential execution on a Intel core i5 4570 processor with 8-GB main memory), the total time demand of annotating bands in all the 95 data sets did not exceed 4 min (for each data set, mean 2.2837 sec; median 2.2707 sec).
%According to the description in \citet{vasa2008shapefinder}, ShapeFinder has a quadratic time complexity. However, ShapeFinder tended to be sensitive to the threshold level set to detect a band position in our experiments and required additional efforts to adjust threshold levels not to miss too many band locations.
\end{comment}